\documentclass[13pt,a4paper,twocolumn,titlepage]{scrartcl}
\usepackage[top=1.5cm, bottom=3cm, left=1.5cm, right=1.5cm]{geometry}
\usepackage{hyperref}
\usepackage[english]{babel}
\addtokomafont{section}{\renewcommand*\familydefault{\rmdefault}\fontsize{30}{30}\setmainfont{QTFraktur}\vspace*{-6mm}}
\addtokomafont{subsection}{\renewcommand*\familydefault{\rmdefault}\fontsize{20}{20}\setmainfont{QTFraktur}\vspace*{-7mm}}
\addtokomafont{paragraph}{\renewcommand*\familydefault{\rmdefault}\fontsize{15}{15}\setmainfont{QTFraktur}}
\usepackage[skip=3mm, indent=0mm]{parskip}

\usepackage[x11names]{xcolor}
\colorlet{fascist}{OrangeRed1}
\colorlet{liberal}{Turquoise4}

\usepackage{yfonts}
\usepackage{fontspec}
\setmainfont{QTHeidelbergType}

\usepackage{tikz}
\usepackage{enumitem}

\begin{document}
	\normalfont
	
	\begin{titlepage}
		\centering
		\topskip0pt
		\vspace*{\fill}
		\fontsize{130}{130}\selectfont SECRET\\HITLER\\
		\vspace{2cm}
		\fontsize{45}{45}\selectfont LaTeX VERSION
		\vspace*{\fill}
	\end{titlepage}

	The year is 1932. The place is pre-WII Germany. In Secret Hitler​, players are German politicians attempting to hold a fragile \textcolor{liberal}{Liberal} government together and stem the rising tide of \textcolor{fascist}{Fascism}. Watch out though—there are secret \textcolor{fascist}{Fascists} among you, and one player is \textcolor{fascist}{Secret Hitler}.
	
	\section*{Overview}
	At the beginning of the game, each player is secretly assigned to one of three roles: 
	\textcolor{liberal}{\textbf{Liberal}}​, \textcolor{fascist}{\textbf{Fascist}​}, or \textcolor{fascist}{​\textbf{Hitler}​}. The \textcolor{liberal}{Liberals} have a majority, but they don’t know for sure who anyone is; \textcolor{fascist}{Fascists} must resort to secrecy and sabotage to accomplish their goals. \textcolor{fascist}{Hitler} plays for the \textcolor{fascist}{Fascist} team, and the \textcolor{fascist}{Fascists} know \textcolor{fascist}{Hitler}’s identity from the outset, but \textcolor{fascist}{Hitler} doesn’t know the \textcolor{fascist}{Fascists} and must work to figure them out.
	
	\textbf{The \textcolor{liberal}{Liberals} win by ​enacting five \textcolor{liberal}{Liberal} Policies​ or killing \textcolor{fascist}{​Hitler}​. The \textcolor{fascist}{​Fascists} win by ​enacting six \textcolor{fascist}{​Fascist​} Policies, or if ​\textcolor{fascist}{​Hitler} is elected Chancellor​ after three \textcolor{fascist}{​Fascist} Policies have been enacted.}
	
	Whenever a \textcolor{fascist}{​Fascist} Policy is enacted, the government becomes more powerful, and the President is granted a single-use power which must be used before the next round can begin. It doesn’t matter what team the President is on; in fact, even \textcolor{liberal}{​Liberal} players might be tempted to enact a \textcolor{fascist}{​Fascist} Policy to gain new powers.
	
	\section*{Object}
	Every player has a secret identity as a member of either the \textcolor{liberal}{​Liberal} team or the \textcolor{fascist}{​Fascist} team.
	
	\textcolor{liberal}{​Players on the Liberal team win if either:
	\begin{itemize}
		\item Five Liberal Policies are enacted.
		\item Hitler is assassinated.
	\end{itemize}
	}

	\textcolor{fascist}{​Players on the Fascist team win if either:
	\begin{itemize}
		\item Six Fascist Policies are enacted.
		\item Hitler is elected Chancellor any time after the third Fascist Policy has been enacted.
	\end{itemize}
	}

	\section*{Game contents}
	\begin{itemize}
		\item \textbf{17} Policy tiles (\textcolor{liberal}{\textbf{6} Liberal}, \textcolor{fascist}{\textbf{11} Fascist})
		\item \textbf{10} Secret Role cards (6 \textcolor{liberal}{Liberal}, 3 \textcolor{fascist}{Fascist}, 1 \textcolor{fascist}{Hitler})
		\item \textbf{10} Party Membership cards (6 \textcolor{liberal}{Liberal}, 4 \textcolor{fascist}{Fascist})
		\item \textbf{10} Card envelopes
		\item \textbf{10} Ja! Ballot cards
		\item \textbf{10} Nein! Ballot cards
		\item \textbf{1} Election Tracker marker
		\item \textbf{1} Draw pile card
		\item \textbf{1} Discard pile card
		\item \textbf{3} Fascist boards
		\item \textbf{1} Liberal board
		\item \textbf{1} President placard
		\item \textbf{1} Chancellor placard
	\end{itemize}
	
	\section*{Set up}
	Select the \textcolor{fascist}{Fascist} track that corresponds to the number of players and place it next to any \textcolor{liberal}{Liberal} track. Shuffle the 11 \textcolor{fascist}{Fascist} Policy tiles and the 6 \textcolor{liberal}{Liberal} Policy tiles into a single Policy deck and place that deck face down on the Draw pile card.
	
	You’ll need an envelope for each player, and each envelope should contain a Secret Role card, a corresponding Party Membership card, one Ja! Ballot card, and one Nein! Ballot card. Use the table below to determine the correct distribution of roles.
	
	\textcolor{liberal}{Liberal} Secret Role cards must always be packed together with a \textcolor{liberal}{Liberal} Party Membership card, and \textcolor{fascist}{Fascist} and \textcolor{fascist}{Hitler} Secret Role cards must always be packed together with a \textcolor{fascist}{Fascist} Party Membership card.	
	\begin{center}
		\begin{tabular}{l|c|c|c|c|c|c}
			\# Players & 5 & 6 & 7 & 8 & 9 & 10\\
			\hline
			\textcolor{liberal}{Liberals} & 3 & 4 & 4 & 5 & 5 & 6\\
			\hline
			\textcolor{fascist}{Fascists} & 1+H & 1+H & 2+H & 2+H & 3+H & 3+H\\	
		\end{tabular}
	\end{center}
	
	Make sure you have the correct number of ordinary \textcolor{fascist}{Fascists} in addition to \textcolor{fascist}{Hitler}!
	
	Once the envelopes have been filled, be sure to shuffle them so each player’s role is a secret! Each player should get one envelope selected at random.
	
	\vspace*{-8mm}
	\paragraph{\textcolor{SeaGreen4}{Why are there secret role and party membership cards?}}
	\textcolor{SeaGreen4}{Secret Hitler features an investigation mechanic that allows some players to find out what team other players are on, and this mechanic only works if Hitler’s special role is not revealed. To prevent that from happening, every player has both a Secret Role card and a Party Membership card. Hitler’s Party Membership card shows a Fascist party loyalty, but gives no hint about a special role. Liberals who uncover Fascists must work out for themselves whether they’ve found an ordinary Fascist or their leader.}
	
	Once each player has been dealt an envelope, all players should examine their Secret Role cards in secret. Randomly select the first Presidential Candidate and pass that player both the President and Chancellor placards.
	
	\vspace*{-8mm}
	\paragraph{\textcolor{SeaGreen4}{Get the App}}
	\textcolor{SeaGreen4}{Go to \url{secrethitler.com/app} to get a companion app that can narrate these directions for you.}
	
	\textcolor{Blue3}{For games of 5-6 players}, give the following directions to all players:
	\begin{itemize}
		\item Everbody close your eyes.
		\item \textcolor{fascist}{Fascist} and \textcolor{fascist}{Hitler}, open your eyes and acknowledge each other.
		\item $\phantom{}$[Take a long pause].
		\item Everyone close your eyes.
		\item Everyone can open your eyes. If anyone is confused or something went wrong, please tell the group now.
	\end{itemize}

		\textcolor{Blue3}{For games of 7-10 players}, give the following directions to all players:
	\begin{itemize}
		\item Everybody close your eyes and extend your hand into a fist in front of you.
		\item All \textcolor{fascist}{Fascists} who are NOT \textcolor{fascist}{Hitler} should open their eyes and acknowledge each other.
		\item \textcolor{fascist}{Hitler} - keep your eyes closed but put your thumb out into a thumbs-up gesture.
		\item \textcolor{fascist}{Fascists}, take note of who has an extended thumb - that player is \textcolor{fascist}{Hitler}.
		\item $\phantom{}$[Take a long pause].
		\item Everbody close your eyes and put your hands down.
		\item Everyone can open your eyes. If anyone is confused or something went wrong, please tell the group now.
	\end{itemize}
	
	\section*{Gameplay}
	Secret Hitler is played in rounds. Each round has an \textcolor{Blue3}{Election} to form a government, a \textcolor{Blue3}{Legislative Session} to enact a new Policy, and an \textcolor{Blue3}{Executive Action} to exercise governmental power.
	\subsection*{Election}
	\begin{enumerate}
		\item \textcolor{Blue3}{Pass the Presidential Candidacy}
		
		At the beginning of a new round, the President placard moves clockwise to the next player, who is the new Presidential Candidate.
		
		\item \textcolor{Blue3}{Nominate a Chancellor}
		
		The Presidential Candidate chooses a Chancellor Candidate by passing the Chancellor placard to any other eligible player. The Presidential Candidate is free to discuss Chancellor options with the table to build consensus and make it more likely the Government gets elected.
		
		\textcolor{Blue3}{Eligibility:} The last elected President and Chancellor are
		“term-limited,” and ineligible to be nominated
		as Chancellor Candidate.
		
		\vspace*{-8mm}
		\paragraph{\textcolor{SeaGreen4}{On Eligibility}}
		\textcolor{SeaGreen4}
		{
			\begin{itemize}
				\item Term limits apply to the President and Chancellor who were last elected, not to the last pair nominated.
				\item Term limits only affect nominations to the Chancellorship; anyone can be President, even	someone who was just Chancellor.
				\item If there are only five players left in the game,​ only the last elected ​Chancellor is ineligible to be Chancellor Candidate; the last President may be nominated.
				\item There are some other rules that affect eligibility in specific ways: the Veto Power and the Election Tracker. You don’t need to worry about those yet, and we’ll talk about each one in its relevant section.
			\end{itemize}
		}
		
		\item \textcolor{Blue3}{Vote on the goverment}
		
		Once the Presidential Candidate has chosen an eligible Chancellor Candidate, players may	discuss the proposed government until everyone is ready to vote. Every player, including the Candidates, votes on the proposed government. Once everyone is ready to vote, reveal your Ballot cards simultaneously so that everyone’s vote is public.
		
		\textcolor{Blue3}{If the vote is a tie, or if a majority of players votes no:} The vote fails. The Presidential Candidate misses this chance to be elected, and the President placard moves clockwise to the next	player. The Election Tracker is advanced by	one Election.
		
		\textcolor{Blue3}{Election Tracker:} If the group rejects three governments in a row, the country is thrown into	chaos. Immediately reveal the Policy on top of the Policy deck and enact it. Any power granted by this Policy is ignored, but the Election Tracker resets, and existing term-limits are forgotten. All players become eligible to hold	the office of Chancellor for the next Election. If there are fewer than three tiles remaining in the Policy deck at this point, shuffle them with the Discard pile to create a new Policy deck.
		
		Any time a new Policy tile is played face-up, the Election Tracker is reset, whether it was enacted by an elected government or enacted by the frustrated populace.
		
		\textcolor{Blue3}{If a majority of players votes yes:}​ The Presidential Candidate and Chancellor Candidate become the new President and Chancellor, respectively. \textcolor{Blue3}{If three or more \textcolor{fascist}{Fascist} Policies have been enacted already:} Ask if the new Chancellor is \textcolor{fascist}{Hitler}. If so, the game is over and the \textcolor{fascist}{Fascists} win. Otherwise, other players know for sure the	Chancellor is not \textcolor{fascist}{Hitler}. Proceed as usual to the Legislative Session.	
	\end{enumerate}
	
	\subsection*{Legislative Session}
	During the Legislative Session, the President and Chancellor work together to enact a new Policy in secret. The President draws the top three tiles from the Policy deck, looks at them in secret, and discards one tile face down into the Discard pile. The remaining two tiles go to the Chancellor, who looks in secret, discards one Policy tile face down, and enacts the	remaining Policy by placing the tile face up on the corresponding track.
	
	Verbal and nonverbal communication between the President and Chancellor is forbidden. The President and Chancellor MAY NOT pick Policies to play at random, shuffle the tiles before discarding one, or do anything else clever to avoid secretly and intentionally selecting a
	Policy. Additionally, the President should hand both Policies over at the same time, rather than one at a time to gauge the Chancellor’s reaction. Attempting to telegraph the contents of your hand using randomness or any other unusual selection procedure violates the spirit of the game. Don’t do it.
	
	\textcolor{Blue3}{Discarded Policy tiles should never be revealed to the group. Players must rely on the word of the President and Chancellor, who are free to lie.}
	
	\vspace*{-8mm}
	\paragraph{\textcolor{SeaGreen4}{About lying:}}
	\textcolor{SeaGreen4}
	{
		Often, some players learn things that the rest of the players don’t know, like when the President and Chancellor get to see Policy tiles, or when a President uses the Investigate power to see someone’s Party Membership card. You can always lie about hidden knowledge in Secret Hitler. The only time players MUST tell the truth is in game-ending, Hitler-related scenarios: a player who is Hitler must say so if assassinated or if elected Chancellor after three Fascist Policies have been enacted.
	}
	
	\textcolor{Blue3}{If there are fewer than three tiles remaining in the Policy deck at the end of a Legislative Session}, shuffle them with the Discard pile to create a new Policy deck. Unused Policy tiles should never be ​revealed, and they should not​​ be simply placed on top of the new Policy deck.
	
	\textcolor{Blue3}{If the government enacted a \textcolor{fascist}{Fascist} Policy that
	covered up a Presidential Power}, the sitting
	President gets to use that power. Proceed to the
	Executive Action.
	
	\textcolor{Blue3}{If the government enacted a \textcolor{liberal}{Liberal} Policy or
	a \textcolor{fascist}{Fascist} Policy that grants no Presidential
	Power}, begin a new round with a new Election.
	
	\subsection*{Executive action}
	If the newly-enacted \textcolor{fascist}{Fascist} Policy grants a Presidential Power, the President ​
	must use it before the next round can begin. Before using a power, the President is free to discuss the issue with other players, but ultimately the	President gets to decide how and when the power is used. Gameplay cannot continue until the President uses the power. Presidential Powers are used only once; they don’t stack or roll over to future turns.
	
	\subsection*{Presidential Powers}
	\begin{itemize}[leftmargin=2.5cm]
		
		\item [\tikz{\begin{scope}[scale=0.75]
				\draw [black, line width=4] (-0.5,0.5) circle (0.6);		
				\fill [black!50!white] (-0.5,0.5) circle (0.35);
				\fill [black] (-0.5,0.5) circle (0.15);
				\draw [black,line width=4] (-0.1,0.1)--(0.2,-0.2);
				\fill [black, rounded corners, rotate=-45] (0.2,0.2) rectangle (2,-0.2); 	
		\end{scope}}] \textcolor{Blue3}{Investigate Loyalty}
		
		 The President chooses a player to 	investigate. Investigated players should hand their Party Membership	card (not Secret Role card!) to the	President, who checks the player’s loyalty in secret and then returns the card to the player. The President may share (or lie about!) the	results of their investigation at their discretion. No player may be investigated twice in the same game.

		\item [\tikz{\begin{scope}[scale=0.7]
				\fill [black, rounded corners=5, rotate=-45] (-0.80,-0.18) rectangle(1.3,0.18);
				\fill [black, rounded corners=5, rotate=-45] (0.5,0.22) rectangle(1.3,0.58);
				\fill [black, rounded corners=5, rotate=-45] (0.55,0.62) rectangle(1.3,0.98);
				\fill [black, rounded corners=5, rotate=-45] (0.6,1.02) rectangle(1.3,1.38);
				\fill [black, rounded corners=5, rotate=-45] (0.85,-0.2) rectangle(2,1.4);
				\fill [black, rounded corners=5, rotate=15] (0.1,-1.8) rectangle(1,-1.4);
		\end{scope}}] \textcolor{Blue3}{Call Special Election}
		
		The President chooses any other	player at the table to be the next Presidential Candidate by passing that player the President placard. Any player can become President—even players that are term-limited. The new President nominates an eligible player as Chancellor Candidate and the Election proceeds as usual.
		
		\textcolor{Blue3}{A Special Election does not skip any players. After a Special Election, the President placard returns to the left of the President who enacted the Special Election.}
		
		If the President passes the	presidency to the next player in the rotation, that player would get to run for President twice in a row: once for the Special Election and once for their normal shift in the	Presidential rotation.
		
		\item [\tikz{\begin{scope}[xshift=-15.35, yshift=-5.3, scale=0.8]
				\fill [black, rounded corners] (13.35,8) rectangle (15.35,8.2);
				\fill [black, rounded corners] (13.35,7.7) rectangle (15.35,7.9);
				\fill [black, rounded corners] (13.35,7.4) rectangle (15.35,7.6);
				\fill [black, rounded corners] (13.35,5.3) rectangle (15.35,7.3);
		\end{scope}	}] \textcolor{Blue3}{Policy Peek}
	
		The President secretly looks at the top three tiles in the Policy deck and then returns them to the top of the deck without changing the order.
		
		\item [\tikz{\begin{scope}[scale=0.7]
				\fill [black, rotate=-45] (0,2.2) ellipse (0.15 and 0.75);
				\fill [black!50!white, rounded corners, rotate=-45] (-0.3,-0.1) rectangle (0.3,0.1);
				\fill [black, rotate=-45] (-0.2,0.05) rectangle (0.2,0.25);
				\fill [black!50!white, rotate=-45, rounded corners] (-0.3,0.25) rectangle (0.3,1.75);
				\fill [black!50!white, rotate=-45] (-0.15,1.75) rectangle (0.15,2.2);
		\end{scope}}] \textcolor{Blue3}{Execution}
	
		The President executes one player at the table by saying “I formally execute [player name].” If that player is \textcolor{fascist}{Hitler}, the game ends in a \textcolor{liberal}{Liberal} victory. If the executed player is not ​\textcolor{fascist}{Hitler}, ​the table should \textcolor{Blue3}{not} learn whether a \textcolor{fascist}{Fascist} or a \textcolor{liberal}{Liberal} has been killed; players must try to work out for themselves the new table balance. Executed	players are removed from the game and may not speak, vote, or run	for office.		
	\end{itemize}

	\subsection*{Veto Power}
	The Veto Power is a special rule that comes into effect after five \textcolor{fascist}{Fascist} Policies have been enacted. For all Legislative Sessions after the fifth \textcolor{fascist}{Fascist} Policy is enacted, the Executive branch gains a permanent new ability to discard all three Policy tiles if \textcolor{Blue3}{both} the Chancellor and President agree.
	
	The President draws three Policy tiles, discards one, and passes the remaining two to the Chancellor as usual. Then Chancellor may, instead of enacting either Policy, say “I wish to veto this agenda.” If the President consents	by saying, “I agree to the veto,” both Policies are discarded and the President placard passes to the left as usual. If the President does not consent, the Chancellor must enact a Policy as normal.
	
	Each use of the Veto Power represents an inactive government and advances the Election Tracker by one.
	
	\section*{Strategy notes}
	\begin{itemize}
		\item \textcolor{Blue3}{Everyone should claim to be a \textcolor{liberal}{Liberal}.} Since the \textcolor{liberal}{Liberal} team has a voting majority, it can easily shut out any player claiming to be a \textcolor{fascist}{Fascist}. As a \textcolor{fascist}{Fascist}, there is no advantage to outing yourself to the majority. Additionally, \textcolor{liberal}{Liberals} should usually tell the truth. \textcolor{liberal}{Liberals} are trying to figure out the game like a puzzle, so lying can put their team at a significant ​disadvantage.
		
		\item \textcolor{Blue3}{If this is your first time playing \textcolor{fascist}{Hitler}, just remember: be as \textcolor{liberal}{Liberal} as possible.} Enact \textcolor{liberal}{Liberal} Policies. Vote for \textcolor{liberal}{Liberal} governments. Kiss babies. Trust your fellow \textcolor{fascist}{Fascists} to create opportunities for you to enact \textcolor{liberal}{Liberal} Policies and to advance \textcolor{fascist}{Fascism} on their turns. The \textcolor{fascist}{Fascists} win by subtly manipulating the table and waiting for the right cover to enact \textcolor{fascist}{Fascist} Policies, not by overtly playing as evil.
		
		\item \textcolor{Blue3}{\textcolor{liberal}{Liberals} frequently benefit from slowing play down and discussing the available information.} \textcolor{fascist}{Fascists} frequently benefit from rushing votes and creating confusion.
		
		\item \textcolor{Blue3}{\textcolor{fascist}{Fascists} most often win by electing \textcolor{fascist}{Hitler}, not by enacting six Policies!} Electing \textcolor{fascist}{Hitler} isn’t an optional or secondary win condition, it’s the core of a successful \textcolor{fascist}{Fascist} strategy. \textcolor{fascist}{Hitler} should always play as a \textcolor{liberal}{Liberal}, and should generally avoid lying or getting into fights and disagreements with other players. When the time comes, \textcolor{fascist}{Hitler} needs the \textcolor{liberal}{Liberals}’ trust to get elected. Even if \textcolor{fascist}{Hitler} isn’t ultimately elected, the distrust sown among \textcolor{liberal}{Liberals} is key to getting \textcolor{fascist}{Fascists} elected late in the game.
		
		\item \textcolor{Blue3}{Ask other players to explain why they took an action.} This is especially important with Presidential Powers—in fact, ask ahead of time whom a candidate is thinking of investigating/appointing/assassinating.
		
		\item \textcolor{Blue3}{If a \textcolor{fascist}{Fascist} Policy comes up, there are only three possible culprits: The President, the Chancellor, or the Policy Deck.} Try to figure out who (or what!) put you in this position.
		
	\end{itemize}
	
	\section*{Thank you}
	\textcolor{Blue3}{Mike, Tommy, and Mac would like to thank:}
	
	Shari Spiro, Dan Shapiro, Elan Lee, Mike Selinker, Luke Crane and everyone else at Kickstarter, everyone who contributed a quote to the Kickstarter, The Cards Against Humanity staff.
	
	Thanks to our early playtesters: Karlee, Tom, Maria, Trin, Andy, Cory, Katie, Veronica, Sandy, Greg, Andrew, and everyone else who helped us refine the earliest versions of the game.
	
	Most especially, thank you to our 34,565 backers who helped us make this game a reality.
	
	\clearpage
	
	Secret Hitler was funded with \textcolor{Blue3}{Kickstarter}.
	\section*{Credits \& License}
	Secret Hitler was created by Mike Boxleiter, Tommy Maranges, and Mac Schubert.
	
	Secret Hitler is licensed under a Creative Commons Attribution-NonCommercial-ShareAlike 4.0 International License.
	
	Secret Hitler LaTeX version was adapted from the \hyperref{https://www.secrethitler.com/}{}{}{original version} by Andrés Laín Sanclemente. All graphical materials were recreated from scratch using LaTeX. The names given to the policies correspond to historial events as presented in \hyperref{https://www.weimarer-republik.net/themenportal/chronik-1918-bis-1933/1932/}{}{}{this chronicle of the Weimar Republic}.
	
	\textcolor{Blue3}{YOU ARE FREE TO:}
	\begin{itemize}
		\item \textcolor{Blue3}{Share} — copy and redistribute the game in any medium or format
		\item \textcolor{Blue3}{Adapt} — remix, transform, and build upon the
		game
	\end{itemize}
	\textcolor{Blue3}{UNDER THE FOLLOWING TERMS:}
	\begin{itemize}
		\item \textcolor{Blue3}{Attribution} — If you make something using our game, you need to give us credit and link back to us, and you need to explain what you changed.
		\item \textcolor{Blue3}{Non-Commercial} — You can’t use our game to	make money.
		\item \textcolor{Blue3}{Share Like} — If you remix, transform, or build upon our game, you have to release your work under the same Creative Commons license	that we use (BY-NC-SA 4.0).
		\item \textcolor{Blue3}{No additional restrictions} — You can’t apply legal terms or technological measures to your work that legally restrict others from doing anything our license allows. That means you can’t submit anything using our game to any app store without our approval.
	\end{itemize}

	You can learn more about Creative Commons at CreativeCommons.org. (Our license is available at \url{https://www.CreativeCommons.org/licenses/by-nc-sa/4.0/legalcode}).
\end{document}